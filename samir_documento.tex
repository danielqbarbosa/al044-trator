\documentclass[12pt,]{article}
\usepackage{lmodern}
\usepackage{amssymb,amsmath}
\usepackage{ifxetex,ifluatex}
\usepackage{fixltx2e} % provides \textsubscript
\ifnum 0\ifxetex 1\fi\ifluatex 1\fi=0 % if pdftex
  \usepackage[T1]{fontenc}
  \usepackage[utf8]{inputenc}
\else % if luatex or xelatex
  \ifxetex
    \usepackage{mathspec}
  \else
    \usepackage{fontspec}
  \fi
  \defaultfontfeatures{Ligatures=TeX,Scale=MatchLowercase}
\fi
% use upquote if available, for straight quotes in verbatim environments
\IfFileExists{upquote.sty}{\usepackage{upquote}}{}
% use microtype if available
\IfFileExists{microtype.sty}{%
\usepackage{microtype}
\UseMicrotypeSet[protrusion]{basicmath} % disable protrusion for tt fonts
}{}
\usepackage[margin=1in]{geometry}
\usepackage{hyperref}
\hypersetup{unicode=true,
            pdftitle={Projeto Trator},
            pdfkeywords={pandoc, r markdown, knitr},
            pdfborder={0 0 0},
            breaklinks=true}
\urlstyle{same}  % don't use monospace font for urls
\usepackage{graphicx,grffile}
\makeatletter
\def\maxwidth{\ifdim\Gin@nat@width>\linewidth\linewidth\else\Gin@nat@width\fi}
\def\maxheight{\ifdim\Gin@nat@height>\textheight\textheight\else\Gin@nat@height\fi}
\makeatother
% Scale images if necessary, so that they will not overflow the page
% margins by default, and it is still possible to overwrite the defaults
% using explicit options in \includegraphics[width, height, ...]{}
\setkeys{Gin}{width=\maxwidth,height=\maxheight,keepaspectratio}
\IfFileExists{parskip.sty}{%
\usepackage{parskip}
}{% else
\setlength{\parindent}{0pt}
\setlength{\parskip}{6pt plus 2pt minus 1pt}
}
\setlength{\emergencystretch}{3em}  % prevent overfull lines
\providecommand{\tightlist}{%
  \setlength{\itemsep}{0pt}\setlength{\parskip}{0pt}}
\setcounter{secnumdepth}{0}
% Redefines (sub)paragraphs to behave more like sections
\ifx\paragraph\undefined\else
\let\oldparagraph\paragraph
\renewcommand{\paragraph}[1]{\oldparagraph{#1}\mbox{}}
\fi
\ifx\subparagraph\undefined\else
\let\oldsubparagraph\subparagraph
\renewcommand{\subparagraph}[1]{\oldsubparagraph{#1}\mbox{}}
\fi

%%% Use protect on footnotes to avoid problems with footnotes in titles
\let\rmarkdownfootnote\footnote%
\def\footnote{\protect\rmarkdownfootnote}

%%% Change title format to be more compact
\usepackage{titling}

% Create subtitle command for use in maketitle
\newcommand{\subtitle}[1]{
  \posttitle{
    \begin{center}\large#1\end{center}
    }
}

\setlength{\droptitle}{-2em}
  \title{Projeto Trator}
  \pretitle{\vspace{\droptitle}\centering\huge}
  \posttitle{\par}
  \author{true}
  \preauthor{\centering\large\emph}
  \postauthor{\par}
  \date{}
  \predate{}\postdate{}


\begin{document}
\maketitle

\section{Projeto Trator}\label{projeto-trator}

\textbf{Sistema de Analise de Maquinas e Implementos Rurais}

\textbf{Planejamento e Analise operacional}

\section{Configuração do ambiente
computacional}\label{configuracao-do-ambiente-computacional}

Procedimento aplicado em ambiente Windows.

\subsection{Linguagem selecionada}\label{linguagem-selecionada}

\begin{itemize}
\tightlist
\item
  Python 3.6 64bit.\footnote{Teste de nota de rodapé}
\item
  Biblioteca Kivy 1.10.1.dev0.\footnote{Kivy
    (\protect\hyperlink{ref-kivy2}{2017})}
\item
  RStudio.
\end{itemize}

\subsection{Justificativa:}\label{justificativa}

\begin{itemize}
\tightlist
\item
  Linguagem Multiplataforma (cross-plataform), podendo ser desenvolvido
  para aplicações desktop e mobile;
\item
  Open Source;
\item
  Possui vários pacotes de ferramentas científicas.
\end{itemize}

\subsection{Requisitos:}\label{requisitos}

\begin{itemize}
\tightlist
\item
  Instalação do Python 3.6;\footnote{(Python
    \protect\hyperlink{ref-python362}{2017}).}
\item
  Instalação do Kivy 1.10.1.
\item
  Ferramenta IDE PyCharm.
\end{itemize}

\subsection{Procedimento de
instalação}\label{procedimento-de-instalacao}

\subsubsection{1- Python versão 3.6.}\label{python-versao-3.6.}

\begin{itemize}
\tightlist
\item
  Download a partir de
  \url{https://www.python.org/downloads/release/python-362/}.
\end{itemize}

Diretório de instalação c:/Opt/Python.

\subsubsection{2- PyCharm! IDE para Python e
Django.}\label{pycharm-ide-para-python-e-django.}

Download a partir de \url{https://www.jetbrains.com/pycharm/download/}

\subsubsection{3 Kivy}\label{kivy}

A biblioteca Kivy é um framework para desenvolvimento multiplataforma,
escrito majoritariamente com a linguagem Python/Cython, permite o
desenvolvimento de aplicações para diversos sistemas operacionais, tais
com, Microsoft Windows, Linux, Mac, Android, iOS, Raspberry utilizando
um mesmo código.

O projeto é composto por vários sub-projetos, cada um especializado numa
determinada tarefa, como por exemplo, a geração de executáveis para
determinada plataforma ou então, uma API genérica para facilmente
acessarmos o hardware em qualquer dispositivos e em diferentes
plataformas sem a necessidade de escrevermos uma única linha de código a
mais.

É importante não confundir o projeto Kivy com a biblioteca Kivy. O
projeto Kivy, cujo site é \url{http://kivy.org} é composto por vários
sub-projetos, dentre estes, a biblioteca Kivy.

\textbf{Dependências do Kivy:}

\begin{itemize}
\tightlist
\item
  gstreamer for audio and video-
  \url{https://gstreamer.freedesktop.org/download/};
\item
  glew and/or angle (3.5 only) for OpenGL-
  \url{http://glew.sourceforge.net/};
\item
  sdl2 for control and/or OpenGL.
\end{itemize}

\textbf{Instalação via `pip' no prompt do DOS:}

Para window versão 32bit:

\begin{itemize}
\tightlist
\item
  python -m pip install kivy;
\end{itemize}

Para windows versão 64bit AMD:

\begin{itemize}
\tightlist
\item
  python -m pip install
  C:/Downloads/Python/Kivy-1.10.1.dev0-cp36-cp36m-win\_amd64.whl;
\end{itemize}

Kivy exemplos (\url{https://kivy.org/downloads/appveyor/kivy/}):

\begin{itemize}
\item
  python -m pip install
  C:/Downloads/Python/Kivy\_examples-1.10.1.dev0.20170930.5f6501fa-py2.py3-none-any.whl
\item
  python -m pip install docutils pygments pypiwin32 kivy.deps.sdl2
  kivy.deps.glew;
\item
  python -m pip install kivy.deps.gstreamer;
\item
  python -m pip install kivy.deps.angle ;
\end{itemize}

\textbf{OpenGL 2.0}

OpenGL é uma API de desenvolvimento de aplicações gráficas e, ao mesmo
tempo, o nome de uma linguagem de programação semelhante ao C++. A
palavra OpenGL é um acrônimo de Open Graphics Library`, que, numa
tradução livre teríamos Biblioteca Gráfica Aberta. O seu uso permite o
fácil desenvolvimento de aplicações gráficas, inclusive com ambientes 3D
que podem ser executado em praticamente todos sistemas operacionais e
nos principais dispositivos. A biblioteca é amplamente utilizada na
construção de jogos, ferramentas 3D ou qualquer aplicação que faça uso
intensivo do hardware gráfico.

\textbf{Bug do OpenGL:}

The problem is that all Windows versions come with OpenGL 1.1 by
default, and the Kivy Framework that runs the Buffered VPN application
needs OpenGL 2.0 support.

Definir uma variável de ambiente no Windows para resolver o bug do Open
GL=versão 1.1:

Solução para o Python 3.6 e windows 8.1. Resolve o problema na maioria
dos casos:

\begin{enumerate}
\def\labelenumi{\arabic{enumi}.}
\tightlist
\item
  right click on This PC then open Properties .
\item
  go to Advanced system settings .
\item
  click on Environment Variables .
\item
  click on New in User variables for --- .
\item
  put KIVY\_GL\_BACKEND in Variable name .
\item
  put angle\_sdl2 in Variable value
\end{enumerate}

Outra solução. Se a solução anterior não funcionar, então se deve
proceder ao upgrade your graphics drivers. For that, you'll need to know
what type of graphics card you have in your system.

\subsubsection{99. Preparação e edição deste
documento}\label{preparacao-e-edicao-deste-documento}

Documento criado na ferramenta RStudio,\footnote{RStudio Team
  (\protect\hyperlink{ref-rstudio}{2015})} usando o formato
Markdown\footnote{Xie
  (\protect\hyperlink{ref-R-bookdown}{2017})\cite{R-bookdown}} e tipo de
documento ``rticles''.

\begin{quote}
install.packages(``rticles'', type = ``source'')
\end{quote}

\subsection*{Referências:}\label{referencias}
\addcontentsline{toc}{subsection}{Referências:}

ACM. Association for Computing Machiney. Advancing Computing as a
Science and Profession.
\url{http://www.acm.org/publications/authors/submissions}.

ACM. Orientaçãoes para Autores sobre o uso adequado.
\url{http://www.acm.org/publications/authors/guidance-for-authors-on-fair-use}

Kivy. Repositório de modelos. Sitio
\url{https://kivy.org/downloads/appveyor/kivy/};

PyCharm! IDE para Python e Django-
\url{https://www.jetbrains.com/pycharm/download/}.

RStudio rticles. \url{https://github.com/rstudio/rticles}.

Formato de citações de publicações usando BibTex.
\url{http://www.acm.org/publications/authors/bibtex-formatting} ;

Markdown. bookdown.
\url{https://bookdown.org/yihui/bookdown/citations.html}

\hypertarget{refs}{}
\hypertarget{ref-kivy2}{}
Kivy. 2017. ``Repositório de Modelos.'' Accessed October 1.
\url{https://kivy.org/downloads/appveyor/kivy/}.

\hypertarget{ref-python362}{}
Python. 2017. ``Python 3.6.2 Is the Second Maintenance Release of Python
3.6.'' Accessed October 1.
\url{https://www.python.org/downloads/release/python-362/}.

\hypertarget{ref-rstudio}{}
RStudio Team. 2015. \emph{RStudio: Integrated Development Environment
for R}. Boston, MA: RStudio, Inc. \url{http://www.rstudio.com/}.

\hypertarget{ref-R-bookdown}{}
Xie, Yihui. 2017. ``Bookdown: Authoring Books and Technical Documents
with R Markdown.'' Accessed October 1.
\url{https://CRAN.R-project.org/package=bookdown}.


\end{document}
